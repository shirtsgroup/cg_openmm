%% Generated by Sphinx.
\def\sphinxdocclass{report}
\documentclass[letterpaper,12pt,english,openany,oneside]{sphinxmanual}
\ifdefined\pdfpxdimen
   \let\sphinxpxdimen\pdfpxdimen\else\newdimen\sphinxpxdimen
\fi \sphinxpxdimen=.75bp\relax

\PassOptionsToPackage{warn}{textcomp}
\usepackage[utf8]{inputenc}
\ifdefined\DeclareUnicodeCharacter
% support both utf8 and utf8x syntaxes
  \ifdefined\DeclareUnicodeCharacterAsOptional
    \def\sphinxDUC#1{\DeclareUnicodeCharacter{"#1}}
  \else
    \let\sphinxDUC\DeclareUnicodeCharacter
  \fi
  \sphinxDUC{00A0}{\nobreakspace}
  \sphinxDUC{2500}{\sphinxunichar{2500}}
  \sphinxDUC{2502}{\sphinxunichar{2502}}
  \sphinxDUC{2514}{\sphinxunichar{2514}}
  \sphinxDUC{251C}{\sphinxunichar{251C}}
  \sphinxDUC{2572}{\textbackslash}
\fi
\usepackage{cmap}
\usepackage[T1]{fontenc}
\usepackage{amsmath,amssymb,amstext}
\usepackage{babel}



\usepackage{times}
\expandafter\ifx\csname T@LGR\endcsname\relax
\else
% LGR was declared as font encoding
  \substitutefont{LGR}{\rmdefault}{cmr}
  \substitutefont{LGR}{\sfdefault}{cmss}
  \substitutefont{LGR}{\ttdefault}{cmtt}
\fi
\expandafter\ifx\csname T@X2\endcsname\relax
  \expandafter\ifx\csname T@T2A\endcsname\relax
  \else
  % T2A was declared as font encoding
    \substitutefont{T2A}{\rmdefault}{cmr}
    \substitutefont{T2A}{\sfdefault}{cmss}
    \substitutefont{T2A}{\ttdefault}{cmtt}
  \fi
\else
% X2 was declared as font encoding
  \substitutefont{X2}{\rmdefault}{cmr}
  \substitutefont{X2}{\sfdefault}{cmss}
  \substitutefont{X2}{\ttdefault}{cmtt}
\fi


\usepackage[Bjarne]{fncychap}
\usepackage{sphinx}

\fvset{fontsize=\small}
\usepackage{geometry}

% Include hyperref last.
\usepackage{hyperref}
% Fix anchor placement for figures with captions.
\usepackage{hypcap}% it must be loaded after hyperref.
% Set up styles of URL: it should be placed after hyperref.
\urlstyle{same}

\usepackage{sphinxmessages}
\setcounter{tocdepth}{1}



\title{Coarse-grained OpenMM Documentation}
\date{Aug 27, 2019}
\release{0.0.1}
\author{Garrett A. Meek\\ \\Research group of Michael R. Shirts\\ \\Dept. of Chemical and Biological Engineering\\University of Colorado Boulder}
\newcommand{\sphinxlogo}{\vbox{}}
\renewcommand{\releasename}{Release}
\makeindex
\begin{document}

\pagestyle{empty}
\sphinxmaketitle
\pagestyle{plain}
\sphinxtableofcontents
\pagestyle{normal}
\phantomsection\label{\detokenize{index::doc}}


This documentation is generated automatically using Sphinx, which reads all docstring-formatted comments from Python functions in the ‘cg\_openmm’ repository.  (See cg\_openmm/doc for Sphinx source files.)


\chapter{Building OpenMM objects for coarse grained modeling}
\label{\detokenize{build:building-openmm-objects-for-coarse-grained-modeling}}\label{\detokenize{build::doc}}

\section{Building an OpenMM System() and Topology()}
\label{\detokenize{build:building-an-openmm-system-and-topology}}
All OpenMM simulations require a \sphinxhref{https://simtk.org/api\_docs/openmm/api4\_1/python/classsimtk\_1\_1openmm\_1\_1openmm\_1\_1System.html}{System()} and a \sphinxhref{https://simtk.org/api\_docs/openmm/api4\_1/python/classsimtk\_1\_1openmm\_1\_1app\_1\_1topology\_1\_1Topology.html}{Topology()}.

Listed below are functions and classes that aid the building of OpenMM System() and Topology() class objects for coarse grained models with user-defined properties:

\phantomsection\label{\detokenize{build:module-build.cg_build}}\index{build.cg\_build (module)@\spxentry{build.cg\_build}\spxextra{module}}\index{build\_system() (in module build.cg\_build)@\spxentry{build\_system()}\spxextra{in module build.cg\_build}}

\begin{fulllineitems}
\phantomsection\label{\detokenize{build:build.cg_build.build_system}}\pysiglinewithargsret{\sphinxcode{\sphinxupquote{build.cg\_build.}}\sphinxbfcode{\sphinxupquote{build\_system}}}{\emph{cgmodel}}{}
Builds an OpenMM \sphinxhref{https://simtk.org/api\_docs/openmm/api4\_1/python/classsimtk\_1\_1openmm\_1\_1openmm\_1\_1System.html}{System()} object, given a CGModel() as input.
\begin{quote}\begin{description}
\item[{Parameters}] \leavevmode
\sphinxstyleliteralstrong{\sphinxupquote{cgmodel}} (\sphinxstyleliteralemphasis{\sphinxupquote{class}}) \textendash{} CGModel() class object

\item[{Returns}] \leavevmode
OpenMM System() object

\end{description}\end{quote}

\end{fulllineitems}

\index{build\_topology() (in module build.cg\_build)@\spxentry{build\_topology()}\spxextra{in module build.cg\_build}}

\begin{fulllineitems}
\phantomsection\label{\detokenize{build:build.cg_build.build_topology}}\pysiglinewithargsret{\sphinxcode{\sphinxupquote{build.cg\_build.}}\sphinxbfcode{\sphinxupquote{build\_topology}}}{\emph{cgmodel}, \emph{use\_pdbfile=False}, \emph{pdbfile=None}}{}
Construct an OpenMM \sphinxhref{https://simtk.org/api\_docs/openmm/api4\_1/python/classsimtk\_1\_1openmm\_1\_1app\_1\_1topology\_1\_1Topology.html}{Topology()} class object for our coarse grained model,
\begin{quote}\begin{description}
\item[{Parameters}] \leavevmode\begin{itemize}
\item {} 
\sphinxstyleliteralstrong{\sphinxupquote{cgmodel}} (\sphinxstyleliteralemphasis{\sphinxupquote{class}}) \textendash{} CGModel() class object

\item {} 
\sphinxstyleliteralstrong{\sphinxupquote{use\_pdbfile}} (\sphinxstyleliteralemphasis{\sphinxupquote{Logical}}) \textendash{} Determines whether or not to use a PDB file in order to generate the Topology().

\end{itemize}

\end{description}\end{quote}

\begin{sphinxadmonition}{warning}{Warning:}
When ‘use\_pdbfile’=True, this function will use the \sphinxhref{https://simtk.org/api\_docs/openmm/api4\_1/python/classsimtk\_1\_1openmm\_1\_1app\_1\_1pdbfile\_1\_1PDBFile.html}{PDBFile()} class object from OpenMM to build the Topology().  In order for this approach to function correctly, the particle names in the PDB file must match the particle names in the coarse grained model.
\end{sphinxadmonition}

\end{fulllineitems}



\section{Other tools for building and verifying the OpenMM System() and Topology()}
\label{\detokenize{build:other-tools-for-building-and-verifying-the-openmm-system-and-topology}}
Shown below are other functions/tools to build and verify the System/Topology:

\phantomsection\label{\detokenize{build:module-build.cg_build}}\index{build.cg\_build (module)@\spxentry{build.cg\_build}\spxextra{module}}\index{add\_force() (in module build.cg\_build)@\spxentry{add\_force()}\spxextra{in module build.cg\_build}}

\begin{fulllineitems}
\phantomsection\label{\detokenize{build:build.cg_build.add_force}}\pysiglinewithargsret{\sphinxcode{\sphinxupquote{build.cg\_build.}}\sphinxbfcode{\sphinxupquote{add\_force}}}{\emph{cgmodel}, \emph{force\_type=None}}{}
Given a ‘cgmodel’ and ‘force\_type’ as input, this function adds
the OpenMM force corresponding to ‘force\_type’ to ‘cgmodel.system’.
\begin{quote}\begin{description}
\item[{Parameters}] \leavevmode\begin{itemize}
\item {} 
\sphinxstyleliteralstrong{\sphinxupquote{cgmodel}} \textendash{} CGModel() class object.

\item {} 
\sphinxstyleliteralstrong{\sphinxupquote{type}} \textendash{} class

\item {} 
\sphinxstyleliteralstrong{\sphinxupquote{force\_type}} (\sphinxhref{https://docs.python.org/3/library/stdtypes.html\#str}{\sphinxstyleliteralemphasis{\sphinxupquote{str}}}) \textendash{} Designates the kind of ‘force’ provided. (Valid options include: “Bond”, “Nonbonded”, “Angle”, and “Torsion”)

\end{itemize}

\item[{Returns}] \leavevmode
CGModel() class object

\item[{Return type}] \leavevmode
class

\item[{Returns}] \leavevmode
An OpenMM \sphinxhref{https://simtk.org/api\_docs/openmm/api4\_1/python/classsimtk\_1\_1openmm\_1\_1openmm\_1\_1Force.html}{Force()} object.

\item[{Return type}] \leavevmode


\end{description}\end{quote}

\end{fulllineitems}

\index{add\_new\_elements() (in module build.cg\_build)@\spxentry{add\_new\_elements()}\spxextra{in module build.cg\_build}}

\begin{fulllineitems}
\phantomsection\label{\detokenize{build:build.cg_build.add_new_elements}}\pysiglinewithargsret{\sphinxcode{\sphinxupquote{build.cg\_build.}}\sphinxbfcode{\sphinxupquote{add\_new\_elements}}}{\emph{cgmodel}}{}
Add coarse grained particle types to OpenMM.
\begin{quote}\begin{description}
\item[{Parameters}] \leavevmode
\sphinxstyleliteralstrong{\sphinxupquote{cgmodel}} (\sphinxstyleliteralemphasis{\sphinxupquote{class}}) \textendash{} CGModel object (contains all attributes for a coarse grained model).

\item[{Returns}] \leavevmode
particle\_list: a list of the particles that were added to OpenMM’s ‘Element’ List.

\item[{Return type}] \leavevmode
\sphinxhref{https://docs.python.org/3/library/stdtypes.html\#list}{list}

\item[{Example}] \leavevmode
\end{description}\end{quote}

\begin{sphinxVerbatim}[commandchars=\\\{\}]
\PYG{g+gp}{\PYGZgt{}\PYGZgt{}\PYGZgt{} }\PYG{n}{particle\PYGZus{}types} \PYG{o}{=} \PYG{n}{add\PYGZus{}new\PYGZus{}elements}\PYG{p}{(}\PYG{n}{cgmodel}\PYG{p}{)}
\end{sphinxVerbatim}

\begin{sphinxadmonition}{warning}{Warning:}
If the particle names were user defined, and any of the names conflict with existing element names in OpenMM, OpenMM will issue an error exit.
\end{sphinxadmonition}

\end{fulllineitems}

\index{get\_num\_forces() (in module build.cg\_build)@\spxentry{get\_num\_forces()}\spxextra{in module build.cg\_build}}

\begin{fulllineitems}
\phantomsection\label{\detokenize{build:build.cg_build.get_num_forces}}\pysiglinewithargsret{\sphinxcode{\sphinxupquote{build.cg\_build.}}\sphinxbfcode{\sphinxupquote{get\_num\_forces}}}{\emph{cgmodel}}{}
Given a CGModel() class object, this function determines how many forces we are including when evaluating the energy.
\begin{quote}\begin{description}
\item[{Parameters}] \leavevmode
\sphinxstyleliteralstrong{\sphinxupquote{cgmodel}} (\sphinxstyleliteralemphasis{\sphinxupquote{class}}) \textendash{} CGModel() class object

\item[{Returns}] \leavevmode
Number of forces

\item[{Return type}] \leavevmode
\sphinxhref{https://docs.python.org/3/library/functions.html\#int}{int}

\end{description}\end{quote}

\end{fulllineitems}

\index{test\_force() (in module build.cg\_build)@\spxentry{test\_force()}\spxextra{in module build.cg\_build}}

\begin{fulllineitems}
\phantomsection\label{\detokenize{build:build.cg_build.test_force}}\pysiglinewithargsret{\sphinxcode{\sphinxupquote{build.cg\_build.}}\sphinxbfcode{\sphinxupquote{test\_force}}}{\emph{cgmodel}, \emph{force}, \emph{force\_type=None}}{}
Given an OpenMM \sphinxhref{https://simtk.org/api\_docs/openmm/api4\_1/python/classsimtk\_1\_1openmm\_1\_1openmm\_1\_1Force.html}{Force()}, this function determines if there are any problems with its configuration.
\begin{quote}\begin{description}
\item[{Parameters}] \leavevmode\begin{itemize}
\item {} 
\sphinxstyleliteralstrong{\sphinxupquote{cgmodel}} (\sphinxstyleliteralemphasis{\sphinxupquote{class}}) \textendash{} CGModel() class object.

\item {} 
\sphinxstyleliteralstrong{\sphinxupquote{force}} (\sphinxstyleliteralemphasis{\sphinxupquote{class}}) \textendash{} An OpenMM Force() object.

\item {} 
\sphinxstyleliteralstrong{\sphinxupquote{force\_type}} (\sphinxhref{https://docs.python.org/3/library/stdtypes.html\#str}{\sphinxstyleliteralemphasis{\sphinxupquote{str}}}) \textendash{} Designates the kind of ‘force’ provided. (Valid options include: “Nonbonded”)

\end{itemize}

\item[{Returns}] \leavevmode
‘success’, a variable designating whether or not the force test passed.

\item[{Return type}] \leavevmode
Logical

\end{description}\end{quote}

\end{fulllineitems}

\index{test\_forces() (in module build.cg\_build)@\spxentry{test\_forces()}\spxextra{in module build.cg\_build}}

\begin{fulllineitems}
\phantomsection\label{\detokenize{build:build.cg_build.test_forces}}\pysiglinewithargsret{\sphinxcode{\sphinxupquote{build.cg\_build.}}\sphinxbfcode{\sphinxupquote{test\_forces}}}{\emph{cgmodel}}{}
Given a cgmodel that contains positions and an
an OpenMM System() object, this function tests 
the forces for cgmodel.system.

More specifically, this function confirms that the
model does not have any “NaN” or unphysically large forces.
\begin{quote}\begin{description}
\item[{Parameters}] \leavevmode\begin{itemize}
\item {} 
\sphinxstyleliteralstrong{\sphinxupquote{cgmodel}} \textendash{} CGModel() class object.

\item {} 
\sphinxstyleliteralstrong{\sphinxupquote{type}} \textendash{} class

\end{itemize}

\item[{Returns}] \leavevmode
success: Indicates if this cgmodel has unphysical forces.

\item[{Return type}] \leavevmode
Logical

\end{description}\end{quote}

\end{fulllineitems}

\index{verify\_system() (in module build.cg\_build)@\spxentry{verify\_system()}\spxextra{in module build.cg\_build}}

\begin{fulllineitems}
\phantomsection\label{\detokenize{build:build.cg_build.verify_system}}\pysiglinewithargsret{\sphinxcode{\sphinxupquote{build.cg\_build.}}\sphinxbfcode{\sphinxupquote{verify\_system}}}{\emph{cgmodel}}{}
Given a CGModel() class object, this function confirms that its OpenMM \sphinxhref{https://simtk.org/api\_docs/openmm/api4\_1/python/classsimtk\_1\_1openmm\_1\_1openmm\_1\_1System.html}{System()} object is configured correctly.
\begin{quote}\begin{description}
\item[{Parameters}] \leavevmode
\sphinxstyleliteralstrong{\sphinxupquote{cgmodel}} (\sphinxstyleliteralemphasis{\sphinxupquote{class}}) \textendash{} CGModel() class object

\end{description}\end{quote}

\begin{sphinxadmonition}{warning}{Warning:}
The function will force an error exit if the system is invalid, and will proceed as normal if the system is valid.
\end{sphinxadmonition}

\end{fulllineitems}

\index{verify\_topology() (in module build.cg\_build)@\spxentry{verify\_topology()}\spxextra{in module build.cg\_build}}

\begin{fulllineitems}
\phantomsection\label{\detokenize{build:build.cg_build.verify_topology}}\pysiglinewithargsret{\sphinxcode{\sphinxupquote{build.cg\_build.}}\sphinxbfcode{\sphinxupquote{verify\_topology}}}{\emph{cgmodel}}{}
Given a coarse grained model that contains a Topology() (cgmodel.topology), this function verifies the validity of the topology.
\begin{quote}\begin{description}
\item[{Parameters}] \leavevmode
\sphinxstyleliteralstrong{\sphinxupquote{cgmodel}} (\sphinxstyleliteralemphasis{\sphinxupquote{class}}) \textendash{} CGModel() class object.

\end{description}\end{quote}

\begin{sphinxadmonition}{warning}{Warning:}
The function will force an error exit if the topology is invalid, and will proceed as normal if the topology is valid.
\end{sphinxadmonition}

\end{fulllineitems}

\index{write\_xml\_file() (in module build.cg\_build)@\spxentry{write\_xml\_file()}\spxextra{in module build.cg\_build}}

\begin{fulllineitems}
\phantomsection\label{\detokenize{build:build.cg_build.write_xml_file}}\pysiglinewithargsret{\sphinxcode{\sphinxupquote{build.cg\_build.}}\sphinxbfcode{\sphinxupquote{write\_xml\_file}}}{\emph{cgmodel}, \emph{xml\_file\_name}}{}
Write an XML-formatted forcefield file for a coarse grained model.
\begin{quote}\begin{description}
\item[{Parameters}] \leavevmode\begin{itemize}
\item {} 
\sphinxstyleliteralstrong{\sphinxupquote{cgmodel}} (\sphinxstyleliteralemphasis{\sphinxupquote{class}}) \textendash{} CGModel() class object.

\item {} 
\sphinxstyleliteralstrong{\sphinxupquote{xml\_file\_name}} (\sphinxhref{https://docs.python.org/3/library/stdtypes.html\#str}{\sphinxstyleliteralemphasis{\sphinxupquote{str}}}) \textendash{} Path to XML output file.

\end{itemize}

\end{description}\end{quote}

\end{fulllineitems}



\chapter{OpenMM simulation tools for coarse grained modeling}
\label{\detokenize{simulation:openmm-simulation-tools-for-coarse-grained-modeling}}\label{\detokenize{simulation::doc}}

\section{Building OpenMM simulation objects}
\label{\detokenize{simulation:building-openmm-simulation-objects}}
OpenMM simulations are propagated using a \sphinxhref{https://simtk.org/api\_docs/openmm/api4\_1/python/classsimtk\_1\_1openmm\_1\_1app\_1\_1simulation\_1\_1Simulation.html}{Simulation()} object.

Shown below are the main tools needed to build OpenMM Simulaton() objects for coarse grained modeling.

\phantomsection\label{\detokenize{simulation:module-simulation.tools}}\index{simulation.tools (module)@\spxentry{simulation.tools}\spxextra{module}}\index{build\_mm\_simulation() (in module simulation.tools)@\spxentry{build\_mm\_simulation()}\spxextra{in module simulation.tools}}

\begin{fulllineitems}
\phantomsection\label{\detokenize{simulation:simulation.tools.build_mm_simulation}}\pysiglinewithargsret{\sphinxcode{\sphinxupquote{simulation.tools.}}\sphinxbfcode{\sphinxupquote{build\_mm\_simulation}}}{\emph{topology}, \emph{system}, \emph{positions}, \emph{temperature=Quantity(value=300.0}, \emph{unit=kelvin)}, \emph{simulation\_time\_step=None}, \emph{total\_simulation\_time=Quantity(value=1.0}, \emph{unit=picosecond)}, \emph{output\_pdb=None}, \emph{output\_data=None}, \emph{print\_frequency=100}, \emph{test\_time\_step=False}}{}
Construct an OpenMM simulation object for our coarse grained model.

topology: OpenMM topology object

system: OpenMM system object

positions: Array containing the positions of all beads
in the coarse grained model
( np.array( ‘num\_beads’ x 3 , ( float * simtk.unit.distance ) )

temperature: Simulation temperature ( float * simtk.unit.temperature )

simulation\_time\_step: Simulation integration time step
( float * simtk.unit.time )

total\_simulation\_time: Total simulation time ( float * simtk.unit.time )

output\_pdb: Name of output file where we will write the coordinates for this
simulation run ( string )

output\_data: Name of output file where we will write the data from this
simulation ( string )

print\_frequency: Number of simulation steps to skip when writing data
to ‘output\_data’ ( integer )

test\_time\_step: Logical variable determining whether or not the user-provided
time step will be tested prior to a full simulation run ( Logical )
Default = False

\end{fulllineitems}

\index{run\_simulation() (in module simulation.tools)@\spxentry{run\_simulation()}\spxextra{in module simulation.tools}}

\begin{fulllineitems}
\phantomsection\label{\detokenize{simulation:simulation.tools.run_simulation}}\pysiglinewithargsret{\sphinxcode{\sphinxupquote{simulation.tools.}}\sphinxbfcode{\sphinxupquote{run\_simulation}}}{\emph{cgmodel}, \emph{output\_directory}, \emph{total\_simulation\_time}, \emph{simulation\_time\_step}, \emph{temperature}, \emph{print\_frequency}, \emph{output\_pdb=None}, \emph{output\_data=None}}{}
Run OpenMM() simulation

cgmodel: CGModel() class object

output\_directory: Output directory within which to place
the output files from this simulation run.

total\_simulation\_time: The total amount of time for which
we will run this simulation

simulation\_time\_step: The time step for the simulation run.

temperature: The temperature for the simulation run.

print\_frequency: The number of steps to take when writing
simulation data to an output file.

\end{fulllineitems}



\section{Building and running Yank replica exchange simulations}
\label{\detokenize{simulation:building-and-running-yank-replica-exchange-simulations}}
The \sphinxhref{http://getyank.org/0.23.4/index.html}{Yank} python package is used to perform replica exchange sampling with OpenMM simulations.

Shown below are the main functions and tools necessary to conduct Yank replica exchange simulations with a coarse grained model in OpenMM.

\phantomsection\label{\detokenize{simulation:module-simulation.rep_exch}}\index{simulation.rep\_exch (module)@\spxentry{simulation.rep\_exch}\spxextra{module}}\index{run\_replica\_exchange() (in module simulation.rep\_exch)@\spxentry{run\_replica\_exchange()}\spxextra{in module simulation.rep\_exch}}

\begin{fulllineitems}
\phantomsection\label{\detokenize{simulation:simulation.rep_exch.run_replica_exchange}}\pysiglinewithargsret{\sphinxcode{\sphinxupquote{simulation.rep\_exch.}}\sphinxbfcode{\sphinxupquote{run\_replica\_exchange}}}{\emph{topology, system, positions, temperature\_list={[}Quantity(value=250.0, unit=kelvin), Quantity(value=260.0, unit=kelvin), Quantity(value=270.0, unit=kelvin), Quantity(value=280.0, unit=kelvin), Quantity(value=290.0, unit=kelvin), Quantity(value=300.0, unit=kelvin), Quantity(value=310.0, unit=kelvin), Quantity(value=320.0, unit=kelvin), Quantity(value=330.0, unit=kelvin), Quantity(value=340.0, unit=kelvin){]}, simulation\_time\_step=None, total\_simulation\_time=Quantity(value=1.0, unit=picosecond), output\_data='output.nc', print\_frequency=100, verbose\_simulation=False, exchange\_attempts=None, test\_time\_step=False}}{}
Construct an OpenMM simulation object for our coarse grained model.
\begin{quote}\begin{description}
\item[{Parameters}] \leavevmode\begin{itemize}
\item {} 
\sphinxstyleliteralstrong{\sphinxupquote{topology}} (\sphinxstyleliteralemphasis{\sphinxupquote{OpenMM Topology}}\sphinxstyleliteralemphasis{\sphinxupquote{(}}\sphinxstyleliteralemphasis{\sphinxupquote{) }}\sphinxstyleliteralemphasis{\sphinxupquote{class object}}) \textendash{} An OpenMM object which contains information about the bonds and constraints in a molecular model

\item {} 
\sphinxstyleliteralstrong{\sphinxupquote{system}} (\sphinxstyleliteralemphasis{\sphinxupquote{OpenMM System}}\sphinxstyleliteralemphasis{\sphinxupquote{(}}\sphinxstyleliteralemphasis{\sphinxupquote{) }}\sphinxstyleliteralemphasis{\sphinxupquote{class object}}) \textendash{} An OpenMM object which contains information about the forces and particle properties in a molecular model

\item {} 
\sphinxstyleliteralstrong{\sphinxupquote{positions}} (\sphinxstyleliteralemphasis{\sphinxupquote{np.array}}\sphinxstyleliteralemphasis{\sphinxupquote{( }}\sphinxstyleliteralemphasis{\sphinxupquote{'num\_beads' x 3}}\sphinxstyleliteralemphasis{\sphinxupquote{ , }}\sphinxstyleliteralemphasis{\sphinxupquote{( }}\sphinxstyleliteralemphasis{\sphinxupquote{float * simtk.unit.distance}}\sphinxstyleliteralemphasis{\sphinxupquote{ ) }}\sphinxstyleliteralemphasis{\sphinxupquote{)}}) \textendash{} Contains the positions for all particles in a model

\item {} 
\sphinxstyleliteralstrong{\sphinxupquote{temperature\_list}} \textendash{} List of temperatures for which to perform replica exchange simulations, default = {[}(300.0 * unit.kelvin).\_\_add\_\_(i * unit.kelvin) for i in range(-20,100,10){]}

\item {} 
\sphinxstyleliteralstrong{\sphinxupquote{simulation\_time\_step}} (\sphinxstyleliteralemphasis{\sphinxupquote{float * simtk.unit}}) \textendash{} Simulation integration time step, default = None

\item {} 
\sphinxstyleliteralstrong{\sphinxupquote{total\_simulation\_time}} (\sphinxstyleliteralemphasis{\sphinxupquote{float * simtk.unit}}) \textendash{} Total simulation time

\item {} 
\sphinxstyleliteralstrong{\sphinxupquote{output\_data}} (\sphinxstyleliteralemphasis{\sphinxupquote{string}}) \textendash{} Name of NETCDF file where we will write data from replica exchange simulations

\end{itemize}

\end{description}\end{quote}
\begin{description}
\item[{replica\_energies: List( List( float * simtk.unit.energy for simulation\_steps ) for num\_replicas )}] \leavevmode
List of dimension num\_replicas X simulation\_steps, which gives the energies for all replicas at all simulation steps

\item[{replica\_positions: np.array( ( float * simtk.unit.positions for num\_beads ) for simulation\_steps )}] \leavevmode
List of positions for all output frames for all replicas

\item[{replica\_state\_indices: np.array( ( float for num\_replicas ) for exchange\_attempts )}] \leavevmode
List of thermodynamic state assignment labels for each replica
during each stage of the replica exchange simulation run.

\item[{temperature\_list: List( float * simtk.unit.kelvin  for num\_replicas )}] \leavevmode
List of the temperatures for each replica.

\end{description}

\end{fulllineitems}

\index{read\_replica\_exchange\_data() (in module simulation.rep\_exch)@\spxentry{read\_replica\_exchange\_data()}\spxextra{in module simulation.rep\_exch}}

\begin{fulllineitems}
\phantomsection\label{\detokenize{simulation:simulation.rep_exch.read_replica_exchange_data}}\pysiglinewithargsret{\sphinxcode{\sphinxupquote{simulation.rep\_exch.}}\sphinxbfcode{\sphinxupquote{read\_replica\_exchange\_data}}}{\emph{system=None}, \emph{topology=None}, \emph{temperature\_list=None}, \emph{output\_data='output.nc'}, \emph{print\_frequency=None}}{}
\end{fulllineitems}

\index{make\_replica\_pdb\_files() (in module simulation.rep\_exch)@\spxentry{make\_replica\_pdb\_files()}\spxextra{in module simulation.rep\_exch}}

\begin{fulllineitems}
\phantomsection\label{\detokenize{simulation:simulation.rep_exch.make_replica_pdb_files}}\pysiglinewithargsret{\sphinxcode{\sphinxupquote{simulation.rep\_exch.}}\sphinxbfcode{\sphinxupquote{make\_replica\_pdb\_files}}}{\emph{topology}, \emph{replica\_positions}}{}
\end{fulllineitems}



\section{Plotting tools}
\label{\detokenize{simulation:plotting-tools}}
Shown below are functions which allow plotting of simulation results.

\phantomsection\label{\detokenize{simulation:module-simulation.rep_exch}}\index{simulation.rep\_exch (module)@\spxentry{simulation.rep\_exch}\spxextra{module}}\index{plot\_replica\_exchange\_energies() (in module simulation.rep\_exch)@\spxentry{plot\_replica\_exchange\_energies()}\spxextra{in module simulation.rep\_exch}}

\begin{fulllineitems}
\phantomsection\label{\detokenize{simulation:simulation.rep_exch.plot_replica_exchange_energies}}\pysiglinewithargsret{\sphinxcode{\sphinxupquote{simulation.rep\_exch.}}\sphinxbfcode{\sphinxupquote{plot\_replica\_exchange\_energies}}}{\emph{replica\_energies}, \emph{temperature\_list}, \emph{simulation\_time\_step}, \emph{steps\_per\_stage=1}, \emph{file\_name='replica\_exchange\_energies.png'}, \emph{legend=True}}{}
\end{fulllineitems}

\index{plot\_replica\_exchange\_summary() (in module simulation.rep\_exch)@\spxentry{plot\_replica\_exchange\_summary()}\spxextra{in module simulation.rep\_exch}}

\begin{fulllineitems}
\phantomsection\label{\detokenize{simulation:simulation.rep_exch.plot_replica_exchange_summary}}\pysiglinewithargsret{\sphinxcode{\sphinxupquote{simulation.rep\_exch.}}\sphinxbfcode{\sphinxupquote{plot\_replica\_exchange\_summary}}}{\emph{replica\_states}, \emph{temperature\_list}, \emph{simulation\_time\_step}, \emph{steps\_per\_stage=1}, \emph{file\_name='replica\_exchange\_state\_transitions.png'}, \emph{legend=True}}{}
\end{fulllineitems}

\phantomsection\label{\detokenize{simulation:module-simulation.tools}}\index{simulation.tools (module)@\spxentry{simulation.tools}\spxextra{module}}\index{plot\_simulation\_data() (in module simulation.tools)@\spxentry{plot\_simulation\_data()}\spxextra{in module simulation.tools}}

\begin{fulllineitems}
\phantomsection\label{\detokenize{simulation:simulation.tools.plot_simulation_data}}\pysiglinewithargsret{\sphinxcode{\sphinxupquote{simulation.tools.}}\sphinxbfcode{\sphinxupquote{plot\_simulation\_data}}}{\emph{simulation\_times}, \emph{y\_data}, \emph{plot\_type=None}}{}
\end{fulllineitems}

\index{plot\_simulation\_results() (in module simulation.tools)@\spxentry{plot\_simulation\_results()}\spxextra{in module simulation.tools}}

\begin{fulllineitems}
\phantomsection\label{\detokenize{simulation:simulation.tools.plot_simulation_results}}\pysiglinewithargsret{\sphinxcode{\sphinxupquote{simulation.tools.}}\sphinxbfcode{\sphinxupquote{plot\_simulation\_results}}}{\emph{simulation\_data\_file}, \emph{plot\_output\_directory}, \emph{simulation\_time\_step}}{}
\end{fulllineitems}



\section{Other simulation tools}
\label{\detokenize{simulation:other-simulation-tools}}
Shown below are other tools which aid the building and verification of OpenMM simulation objects.

\phantomsection\label{\detokenize{simulation:module-simulation.tools}}\index{simulation.tools (module)@\spxentry{simulation.tools}\spxextra{module}}\index{get\_mm\_energy() (in module simulation.tools)@\spxentry{get\_mm\_energy()}\spxextra{in module simulation.tools}}

\begin{fulllineitems}
\phantomsection\label{\detokenize{simulation:simulation.tools.get_mm_energy}}\pysiglinewithargsret{\sphinxcode{\sphinxupquote{simulation.tools.}}\sphinxbfcode{\sphinxupquote{get\_mm\_energy}}}{\emph{topology}, \emph{system}, \emph{positions}}{}
Get the OpenMM potential energy for a system, given a topology and set of positions.

topology: OpenMM topology object

system: OpenMM system object

positions: Array containing the positions of all beads
in the coarse grained model
( np.array( ‘num\_beads’ x 3 , ( float * simtk.unit.distance ) )

\end{fulllineitems}

\index{get\_simulation\_time\_step() (in module simulation.tools)@\spxentry{get\_simulation\_time\_step()}\spxextra{in module simulation.tools}}

\begin{fulllineitems}
\phantomsection\label{\detokenize{simulation:simulation.tools.get_simulation_time_step}}\pysiglinewithargsret{\sphinxcode{\sphinxupquote{simulation.tools.}}\sphinxbfcode{\sphinxupquote{get\_simulation\_time\_step}}}{\emph{topology}, \emph{system}, \emph{positions}, \emph{temperature}, \emph{total\_simulation\_time}, \emph{time\_step\_list=None}}{}
Determine a suitable simulation time step.
\begin{quote}\begin{description}
\item[{Parameters}] \leavevmode\begin{itemize}
\item {} 
\sphinxstyleliteralstrong{\sphinxupquote{topology}} (\sphinxhref{https://simtk.org/api\_docs/openmm/api4\_1/python/classsimtk\_1\_1openmm\_1\_1app\_1\_1topology\_1\_1Topology.html}{Topology()}) \textendash{} OpenMM Topology

\item {} 
\sphinxstyleliteralstrong{\sphinxupquote{system}} (\sphinxhref{https://simtk.org/api\_docs/openmm/api4\_1/python/classsimtk\_1\_1openmm\_1\_1openmm\_1\_1System.html}{System()}) \textendash{} OpenMM System()

\item {} 
\sphinxstyleliteralstrong{\sphinxupquote{positions}} ({\color{red}\bfseries{}{}`}unit.Quantity() \textless{}\sphinxurl{http://docs.openmm.org/development/api-python/generated/simtk.unit.quantity.Quantity.html}\textgreater{}{}`\_(np.array({[}cgmodel.num\_beads,3{]}),simtk.unit)) \textendash{} Positions array for the model we would like to test

\item {} 
\sphinxstyleliteralstrong{\sphinxupquote{temperature}} (\sphinxhref{https://simtk.org/}{SIMTK} \sphinxhref{http://docs.openmm.org/7.1.0/api-python/generated/simtk.unit.unit.Unit.html}{Unit()}) \textendash{} Simulation temperature

\item {} 
\sphinxstyleliteralstrong{\sphinxupquote{total\_simulation\_time}} \textendash{} Total run time for individual simulations

\item {} 
\sphinxstyleliteralstrong{\sphinxupquote{time\_step\_list}} (\sphinxstyleliteralemphasis{\sphinxupquote{List}}\sphinxstyleliteralemphasis{\sphinxupquote{, }}\sphinxstyleliteralemphasis{\sphinxupquote{default = None}}) \textendash{} List of time steps for which to attempt a simulation in OpenMM.

\end{itemize}

\item[{Returns}] \leavevmode
A successfully-tested time step

\item[{Return type}] \leavevmode

\sphinxhref{https://simtk.org/}{SIMTK} \sphinxhref{http://docs.openmm.org/7.1.0/api-python/generated/simtk.unit.unit.Unit.html}{Unit()}


\end{description}\end{quote}

\end{fulllineitems}

\index{minimize\_structure() (in module simulation.tools)@\spxentry{minimize\_structure()}\spxextra{in module simulation.tools}}

\begin{fulllineitems}
\phantomsection\label{\detokenize{simulation:simulation.tools.minimize_structure}}\pysiglinewithargsret{\sphinxcode{\sphinxupquote{simulation.tools.}}\sphinxbfcode{\sphinxupquote{minimize\_structure}}}{\emph{topology}, \emph{system}, \emph{positions}, \emph{temperature=Quantity(value=0.0}, \emph{unit=kelvin)}, \emph{simulation\_time\_step=None}, \emph{total\_simulation\_time=Quantity(value=1.0}, \emph{unit=picosecond)}, \emph{output\_pdb=None}, \emph{output\_data=None}, \emph{print\_frequency=1}}{}
Perform a minimization of the potential energy for our coarse grained model.
\begin{quote}\begin{description}
\item[{Parameters}] \leavevmode\begin{itemize}
\item {} 
\sphinxstyleliteralstrong{\sphinxupquote{topology}} (\sphinxstyleliteralemphasis{\sphinxupquote{Topology}}\sphinxstyleliteralemphasis{\sphinxupquote{(}}\sphinxstyleliteralemphasis{\sphinxupquote{)}}) \textendash{} OpenMM topology for the coarse grained model

\item {} 
\sphinxstyleliteralstrong{\sphinxupquote{system}} (\sphinxstyleliteralemphasis{\sphinxupquote{System}}\sphinxstyleliteralemphasis{\sphinxupquote{(}}\sphinxstyleliteralemphasis{\sphinxupquote{)}}) \textendash{} OpenMM system

\item {} 
\sphinxstyleliteralstrong{\sphinxupquote{positions}} ({\color{red}\bfseries{}{}`}unit.Quantity() \textless{}\sphinxurl{http://docs.openmm.org/development/api-python/generated/simtk.unit.quantity.Quantity.html}\textgreater{}{}`\_(np.array({[}cgmodel.num\_beads,3{]}),simtk.unit)) \textendash{} Positions array for the model we would like to test

\item {} 
\sphinxstyleliteralstrong{\sphinxupquote{temperature}} \textendash{} Simulation temperature

\item {} 
\sphinxstyleliteralstrong{\sphinxupquote{total\_simulation\_time}} \textendash{} Total run time for individual simulations

\item {} 
\sphinxstyleliteralstrong{\sphinxupquote{output\_pdb}} (\sphinxhref{https://docs.python.org/3/library/stdtypes.html\#str}{\sphinxstyleliteralemphasis{\sphinxupquote{str}}}) \textendash{} Output destinaton for PDB-formatted coordinates during the simulation

\item {} 
\sphinxstyleliteralstrong{\sphinxupquote{output\_data}} (\sphinxhref{https://docs.python.org/3/library/stdtypes.html\#str}{\sphinxstyleliteralemphasis{\sphinxupquote{str}}}) \textendash{} Output destination for simulation data

\item {} 
\sphinxstyleliteralstrong{\sphinxupquote{print\_frequency}} (\sphinxhref{https://docs.python.org/3/library/functions.html\#int}{\sphinxstyleliteralemphasis{\sphinxupquote{int}}}) \textendash{} Number of simulation steps to skip when writing data

\end{itemize}

\item[{Returns}] \leavevmode
positions: Minimized positions

\item[{Return type}] \leavevmode
positions: {\color{red}\bfseries{}{}`}unit.Quantity() \textless{}\sphinxurl{http://docs.openmm.org/development/api-python/generated/simtk.unit.quantity.Quantity.html}\textgreater{}{}`\_(np.array({[}cgmodel.num\_beads,3{]}),simtk.unit)

\item[{Returns}] \leavevmode
potential\_energy: Potential energy for the minimized structure.

\item[{Return type}] \leavevmode
potential\_energy: {\color{red}\bfseries{}{}`}unit.Quantity() \textless{}\sphinxurl{http://docs.openmm.org/development/api-python/generated/simtk.unit.quantity.Quantity.html}\textgreater{}{}`\_(np.array({[}cgmodel.num\_beads,3{]}),simtk.unit)

\end{description}\end{quote}

\end{fulllineitems}

\phantomsection\label{\detokenize{simulation:module-simulation.rep_exch}}\index{simulation.rep\_exch (module)@\spxentry{simulation.rep\_exch}\spxextra{module}}\index{get\_minimum\_energy\_pose() (in module simulation.rep\_exch)@\spxentry{get\_minimum\_energy\_pose()}\spxextra{in module simulation.rep\_exch}}

\begin{fulllineitems}
\phantomsection\label{\detokenize{simulation:simulation.rep_exch.get_minimum_energy_pose}}\pysiglinewithargsret{\sphinxcode{\sphinxupquote{simulation.rep\_exch.}}\sphinxbfcode{\sphinxupquote{get\_minimum\_energy\_pose}}}{\emph{topology}, \emph{replica\_energies}, \emph{replica\_positions}, \emph{file\_name=None}}{}
\end{fulllineitems}



\chapter{Utilities for coarse grained modeling in OpenMM}
\label{\detokenize{utilities:utilities-for-coarse-grained-modeling-in-openmm}}\label{\detokenize{utilities::doc}}
This page details the functionality of utilities in cg\_openmm/src/utilities/util.py.

\phantomsection\label{\detokenize{utilities:module-utilities.util}}\index{utilities.util (module)@\spxentry{utilities.util}\spxextra{module}}\index{distance() (in module utilities.util)@\spxentry{distance()}\spxextra{in module utilities.util}}

\begin{fulllineitems}
\phantomsection\label{\detokenize{utilities:utilities.util.distance}}\pysiglinewithargsret{\sphinxcode{\sphinxupquote{utilities.util.}}\sphinxbfcode{\sphinxupquote{distance}}}{\emph{positions\_1}, \emph{positions\_2}}{}
Construct a matrix of the distances between all particles.

positions\_1: Positions for a particle
( np.array( length = 3 ) )

positions\_2: Positions for a particle
( np.array( length = 3 ) )

distance
( float * unit )

\end{fulllineitems}

\index{get\_box\_vectors() (in module utilities.util)@\spxentry{get\_box\_vectors()}\spxextra{in module utilities.util}}

\begin{fulllineitems}
\phantomsection\label{\detokenize{utilities:utilities.util.get_box_vectors}}\pysiglinewithargsret{\sphinxcode{\sphinxupquote{utilities.util.}}\sphinxbfcode{\sphinxupquote{get\_box\_vectors}}}{\emph{box\_size}}{}
Assign all side lengths for simulation box.

box\_size: Simulation box length ( float * simtk.unit.length )

\end{fulllineitems}

\index{lj\_v() (in module utilities.util)@\spxentry{lj\_v()}\spxextra{in module utilities.util}}

\begin{fulllineitems}
\phantomsection\label{\detokenize{utilities:utilities.util.lj_v}}\pysiglinewithargsret{\sphinxcode{\sphinxupquote{utilities.util.}}\sphinxbfcode{\sphinxupquote{lj\_v}}}{\emph{positions\_1}, \emph{positions\_2}, \emph{sigma}, \emph{epsilon}}{}
Given two sets of input coordinates, this function computes
their Lennard-Jones interaction potential energy.

positions\_1

\end{fulllineitems}

\index{set\_box\_vectors() (in module utilities.util)@\spxentry{set\_box\_vectors()}\spxextra{in module utilities.util}}

\begin{fulllineitems}
\phantomsection\label{\detokenize{utilities:utilities.util.set_box_vectors}}\pysiglinewithargsret{\sphinxcode{\sphinxupquote{utilities.util.}}\sphinxbfcode{\sphinxupquote{set\_box\_vectors}}}{\emph{system}, \emph{box\_size}}{}
Build a simulation box.

system: OpenMM system object

box\_size: Simulation box length ( float * simtk.unit.length )

\end{fulllineitems}



\chapter{Indices and tables}
\label{\detokenize{index:indices-and-tables}}\begin{itemize}
\item {} 
\DUrole{xref,std,std-ref}{genindex}

\item {} 
\DUrole{xref,std,std-ref}{modindex}

\item {} 
\DUrole{xref,std,std-ref}{search}

\end{itemize}


\renewcommand{\indexname}{Python Module Index}
\begin{sphinxtheindex}
\let\bigletter\sphinxstyleindexlettergroup
\bigletter{b}
\item\relax\sphinxstyleindexentry{build.cg\_build}\sphinxstyleindexpageref{build:\detokenize{module-build.cg_build}}
\indexspace
\bigletter{s}
\item\relax\sphinxstyleindexentry{simulation.rep\_exch}\sphinxstyleindexpageref{simulation:\detokenize{module-simulation.rep_exch}}
\item\relax\sphinxstyleindexentry{simulation.tools}\sphinxstyleindexpageref{simulation:\detokenize{module-simulation.tools}}
\indexspace
\bigletter{u}
\item\relax\sphinxstyleindexentry{utilities.util}\sphinxstyleindexpageref{utilities:\detokenize{module-utilities.util}}
\end{sphinxtheindex}

\renewcommand{\indexname}{Index}
\printindex
\end{document}